\begin{vdmpp}[breaklines=true]
class AccessRule

-- related to which Entry point
-- affects which User
-- 
types



public AccessEvent = <YouShallNotPass> | <EnterMountDoom> | <ThatIDoNotKnow>;

public Target :: users : set of User
         facilities : set of Facility
         timeRestriction : set of Date;
         

values
-- TODO Define values here
instance variables


 public userTarget : set of User;
 public facilityTarget : set of Facility;
 public accessEve : AccessEvent := <ThatIDoNotKnow>;
 public dateBegin : Date := new Date(0,0,0,"Indefinite", 0, 0);
 public dateEnd : Date:= new Date(0,0,0,"Indefinite", 0, 0);
 public priority : nat;
 public temporalConstraint : bool := false;


operations

-- FALTA id que identifique a regra para poder incluir excecoes relativas a uma regra. 
-- FALTA int de permissao. 
-- E isto e tudo indispensavel
  
--public AccessRule: (User|User`userGroup) * (Facility|seq of char) ==> ()
(*@
\label{AccessRule:38}
@*)
public AccessRule: set of User * set of Facility * AccessEvent * nat ==> AccessRule
 AccessRule(u,f,a, pri) ==(
 userTarget := u;
 facilityTarget := f;
 accessEve := a;
 priority := pri;
 return self
 );
 
 public AccessRule: set of User * set of Facility *Date* AccessEvent * nat ==> AccessRule
 AccessRule(u,f,d,a, pri) ==(
 userTarget := u;
 facilityTarget := f;
 dateBegin := d;
 dateEnd := d;
 accessEve := a;
 priority := pri;
 return self
 );
 
public AccessRule: set of User * set of Facility * Date * Date * AccessEvent * nat ==> AccessRule
 AccessRule(u,f, d1, d2, a, pri) ==(
 userTarget := u;
 facilityTarget := f;
 accessEve := a;
 dateBegin := d1;
 dateEnd := d2;
 priority := pri;
 temporalConstraint := true;
 return self
 );

(*@
\label{getUserTarget:70}
@*)
public getUserTarget: () ==> set of User 
 getUserTarget() ==(
 return userTarget;
 );
 
(*@
\label{searchUserInTarget:75}
@*)
public searchUserInTarget: User ==> bool
 searchUserInTarget(u) == (
  if (u in set userTarget)
   then return true
   else return false
 );
 
(*@
\label{getPriority:82}
@*)
public getPriority: () ==> nat
 getPriority() == (
 return priority
 );
 
(*@
\label{searchFacilityInTarget:87}
@*)
public searchFacilityInTarget: Facility ==> bool
 searchFacilityInTarget(f) ==(
  if(f in set facilityTarget)
   then return true
   else return false
 );
-- public addExceptions;
-- ADD USER / FACILITY TO RULE
-- REMOVE USER / FACILITY FROM RULE


-- Compara o dia de semana de abertura e hora de fecho com a data actual. 
(*@
\label{isOnOpeningHours:99}
@*)
-- Utiliza os Ids atribuidos aos dias de semana para o fazer, {Segunda,..,Domingo} e {1,...,7}. 
-- Se dia de abertura tem um id > que id de fecho faz o complementar
-- Acrescentar dcl dlc dcl.
 public isOnOpeningWeekDays : Date ==> bool
  isOnOpeningWeekDays(currentDate) == (
  if(dateBegin.getNameDayId()=0 or dateEnd.getNameDayId()=0)
  then return true
  else
  if(dateBegin.getNameDayId() > dateEnd.getNameDayId())
   then return (currentDate.getNameDayId() >= dateBegin.getNameDayId() or currentDate.getNameDayId() <= dateEnd.getNameDayId())
   else return (currentDate.getNameDayId() >= dateBegin.getNameDayId() and currentDate.getNameDayId() <= dateEnd.getNameDayId())
  ); 

(*@
\label{isOnOpeningWeekDays:112}
@*)
--mudar o lugar desta fun��es, fi-la da maneira errada
 public isInTimeInterval : Date ==> bool
  isInTimeInterval(currentDate) == (
  
  dcl hasStarted : bool := false;
  dcl hasEnded : bool := false;
  dcl onTime : bool := false; 
   
  if ((currentDate.year < dateBegin.year)or
   (currentDate.year = dateBegin.year and currentDate.month < dateBegin.month) or
    currentDate.year = dateBegin.year and currentDate.month = dateBegin.month and
(*@
\label{isInTimeInterval:123}
@*)
    currentDate.day < dateBegin.day)
    then hasStarted := true
   else hasStarted := false;
   
  if ((currentDate.year > dateEnd.year)or
  (currentDate.year = dateEnd.year and currentDate.month > dateEnd.month) or
   currentDate.year = dateEnd.year and currentDate.month = dateEnd.month and
   currentDate.day > dateEnd.day)
   then hasEnded := false
  else hasEnded := true;
  
  if (hasStarted and hasEnded or dateBegin.year = 0 or dateBegin  =  dateEnd)
   then onTime := true;
  
  return onTime;
  );
  
 
-- constructor, initializes the Rule

functions
-- TODO Define functiones here
traces
-- TODO Define Combinatorial Test Traces here
end AccessRule
\end{vdmpp}
