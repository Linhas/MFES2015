\begin{vdmpp}[breaklines=true]
class TestPACS is subclass of GeneralTestCases

operations
--Tests For Date
--Faltam testes relacionados com o dia de semana por exemplo
(*@
\label{testDate:6}
@*)
  public testDate: () ==> ()
   testDate() == (
    dcl p: Date := new Date(20, 12, 2016);
    dcl l: Date := new Date(26, 12, 2016, "Monday", 9, 30);
    dcl fev: Date := new Date(27, 2, 2012);
    dcl an: Date := new Date(23, 4, 2012);
    
    assertEqual(20, p.day);
    assertEqual(12, p.month);
    assertEqual(2016, p.year);
    assertEqual(27, fev.day);
    assertEqual(4, an.month);
    
    
    assertEqual(26, l.day);
    assertEqual(12, l.month);
    assertEqual(2016, l.year);
    assertEqual("Monday", l.getNameDay());
    assertEqual(1, l.getNameDayId());
    assertEqual(0, p.getNameDayId());
    assertEqual(9, l.hour);
    assertEqual(30, l.minute);
(*@
\label{testUser:28}
@*)

   );
   
--Tests For User
 public testUser: () ==> ()
  testUser() == (
  dcl u: User := new User("Anna");
  dcl us: User := new User("MecStudent");
  dcl ua: User := new User("InfStudent");
  dcl usq : User := new User("Student");
  
  u.addGroupToUser(us);
  us.addGroupToUser(usq);
  
  assertEqual("Anna", u.getName());
  assertTrue(u.searchGroupInUser(us));
  assertFalse(u.searchGroupInUser(ua));
  u.removeGroupFromUser(us);
  assertFalse(u.searchGroupInUser(us));
  --assertTrue(u.searchGroupInUser(usq));
(*@
\label{testFacility:48}
@*)
  
  );
  
 --Tests For Facility
 public testFacility: () ==> ()
  testFacility() == (
  dcl f: Facility := new Facility("DEI");
  dcl f2: Facility := new Facility("DEE");
  
  dcl a: Facility := new Facility("wc");
  dcl b: Facility := new Facility("I003");
  dcl c: Facility := new Facility("RCOM");
  
  f.addSectionFacility(a);
  f.addSectionFacility(b);
  f.addSectionFacility(c);
  
  f2.addSectionFacility(a);
  
  f.addFacilityToGroup("Department");
  f2.addFacilityToGroup("Department");
  
  assertTrue(f.searchSectionInFacility(a));
  assertFalse(f2.searchSectionInFacility(b));
  
  assertTrue(f.searchGroupInFacility("Department"));
  assertFalse(f.searchGroupInFacility("Outside"));
   
(*@
\label{testAccessCard:76}
@*)
  );
  
   --Tests for AccessCard
  --um expirado, um sem expira��o que depois ganha expira��o, um n�o expirado
  public testAccessCard: () ==> ()
   testAccessCard() == (
   dcl currentDate: Date := new Date(26, 12, 2016, "Monday", 9, 30);
   dcl expdate1: Date := new Date(20, 12, 2016);
   dcl expdate2: Date := new Date(20, 12, 2017);
   dcl u: User := new User("Anakin");
   dcl ug: User := new User("Padme");
   dcl ul: User := new User("Darth Vader");
   
   dcl card1: AccessCard := new AccessCard (u, expdate1);
   dcl card2: AccessCard := new AccessCard(ug);
   dcl card3: AccessCard := new AccessCard(ul, expdate2);
   
   assertTrue(card1.isExpired(currentDate));
   assertFalse(card2.isExpired(currentDate));
   assertFalse(card3.isExpired(currentDate));

   assertEqual("Anakin", card1.getUserOfCard().getName());
  
   card2.setExpirationDate(expdate1);
   assertEqual(expdate1, card2.getExpirationDate());
   assertTrue(card2.isExpired(currentDate));
   
   assertEqual(expdate2, card3.getExpirationDate());
   assertEqual(2, card2.getIdCardOfUser());
   
   card1.removeExpDate();      
(*@
\label{testPhySystem:107}
@*)
   assertFalse(card1.isExpired(currentDate));
   
   card1.changeUserofCard(ul);
   assertEqual("Darth Vader", card1.getUserOfCard().getName());
   
   assertEqual("Darth Vader", card3.getUserOfCard().getName());
   );
   
public testPhySystem: () ==> ()
   testPhySystem() == (
   dcl currentDate: Date := new Date(26, 12, 2016, "Wednesday", 9, 30);
   dcl expdate1: Date := new Date(20, 12, 2016);
   dcl expdate2: Date := new Date(20, 12, 2017);
   dcl ug: User := new User("Padme");
   dcl ul: User := new User("Darth Vader");
   dcl as: User := new User("Obi");
   
   dcl dateTemporalRestraint1: Date := new Date("Tuesday", 9,30);
   dcl dateTemporalRestraint2: Date := new Date("Friday", 19,00);
   dcl christmas: Date := new Date(25, 12, 2016);
   
   dcl card1: AccessCard := new AccessCard(as, expdate1);
   dcl card2: AccessCard := new AccessCard(ug);
   dcl card3: AccessCard := new AccessCard(ul, expdate2);
   
   
  dcl f: Facility := new Facility("DEI");
  
  dcl a: Facility := new Facility("wc");
  dcl b: Facility := new Facility("I003");
  dcl c: Facility := new Facility("RCOM");
  
  dcl sys: PhyAccControlSystem := new PhyAccControlSystem(currentDate);
 
  assertEqual(0, sys.getSizeAccessPolicy());
  
  -- testa se cart�o expirado
  
  assertFalse(sys.requestAccess(card1,a)); 
  sys.createRule({ug, as}, {a}, <EnterMountDoom> , 13);
  assertFalse(sys.requestAccess(card1,a));
  assertTrue(sys.requestAccess(card2,a));
  -- testa caso simples
  sys.createRule({ug}, {a}, <YouShallNotPass> , 12);
   
(*@
\label{testAll:152}
@*)


  assertFalse(sys.requestAccess(card2,a)); 

  --com restricao de tempo, para uma data que se repete
  sys.createRule({ul}, {a}, <YouShallNotPass> , 13);
  sys.createRule({ul}, {a},  dateTemporalRestraint1, dateTemporalRestraint2, <EnterMountDoom> ,10);
  assertTrue(sys.requestAccess(card3,a)); 
    
  --com restricao de tempo, para um dia unico
  sys.createRule({ul}, {f}, christmas, <YouShallNotPass> , 9);
  assertTrue(sys.requestAccess(card3,a)); 
  sys.changeCurrentDate(christmas);
  assertFalse(sys.requestAccess(card3,a)); 
  assertEqual(7, sys.sizeOfLog());  
   );
   
  
   
 -- TestAll like the name indicates, tests all tests and buuuurns (mainly to check code coverage
  public testAll: () ==> ()
   testAll() == (
   testDate();
   testUser();
   testFacility();
   testAccessCard();
   testPhySystem();
   
   );  
   
end TestPACS
\end{vdmpp}
