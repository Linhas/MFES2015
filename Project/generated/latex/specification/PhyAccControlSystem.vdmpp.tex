\begin{vdmpp}[breaklines=true]
class PhyAccControlSystem
/*
  Contains the core model of the Physical Access Control System.
  Defines the variables and the operations available to the Control System Operator.
  FEUP, MFES, 2015/16.
*/
types
--Change to DenyAccess, Authorized, Indeterminate
AccessEvent = <YouShallNotPass> | <EnterMountDoom> | <ThatIDoNotKnow>;



values

instance variables
 public resulList : seq of AccessRule := [];
 public setList : seq of AccessRule := [];
 public currentDate : Date;
 
 public x: nat := 1;
 public y: nat := len resulList;
 --private PolicyEntryPoints: map String to set of Rule := {|->};
 
 -- precisamos de todos os users para aceder a um em particular, cria-se um user, v�-se j� existe
 private usersList: set of User := { };
 
 --precisamos do allCards para guardar todos, para encontrar um user e etc
 private allCards: set of AccessCard := {};
 
 --facto: o set of access rules e uma access policy 
 public accessPolicy : seq of AccessRule := [];

  
 -- O que o log precisa no minimo, fazer um print aceit�vel disto
 private log : seq of AccessLogInfo := [];

 
operations

(*@
\label{PhyAccControlSystem:40}
@*)
 public PhyAccControlSystem : Date ==> PhyAccControlSystem
  PhyAccControlSystem(d) == (
  currentDate := d;
  -- return self
  );
  -- Makes request sentence
(*@
\label{requestAccess:46}
@*)
 public requestAccess: AccessCard * Facility ==> bool
  requestAccess(c,f) ==(
   dcl enter : bool;
   dcl info : AccessLogInfo;
   dcl l :  AccessRule`AccessEvent;
   if c.AccessCard`isExpired(currentDate)
    then enter := false
    else (

   l :=  findRuleApplicable(c.getUserOfCard(), f, currentDate);
   if l = <ThatIDoNotKnow> or l = <YouShallNotPass>   then  enter:= false
   else if l = <EnterMountDoom> then  enter:= true;);
   info := new AccessLogInfo(f,c,currentDate, enter);
(*@
\label{sizeOfLog:59}
@*)
   log := [info] ^ log;
   
   return enter;
  );  
 public sizeOfLog : () ==> nat
(*@
\label{printLog:64}
@*)
  sizeOfLog() == (
   return len log
  );

 public findRuleApplicable: User * Facility * Date ==> AccessRule`AccessEvent
  findRuleApplicable(u, f, d) == (

   dcl rule : AccessRule;
   dcl ret :  AccessRule`AccessEvent;
   resulList := [];
   setList := accessPolicy;
   filterByUserList(u, setList, resulList);
   setList := resulList;
   resulList := [];
   filterByFacilityList(f, setList, resulList);
   y := len resulList;
   x := 1;
   orderByPermission(); --permission mais baixa mais autoridade
   setList := resulList;
   resulList := [];
   findRuleOnTime(d, setList, resulList); 
  if(resulList = [])
   then ret := <ThatIDoNotKnow>
   else (rule := hd resulList;

   ret := rule.accessEve;);

(*@
\label{findRuleApplicable:91}
@*)
   return ret
  );
  
   public findRuleOnTime: Date * seq of AccessRule * seq of AccessRule ==> () --|seq of AccessRule 
   findRuleOnTime(d, setL, resulL) == (
    dcl a: AccessRule;
    resulList := resulL; 
    setList := setL;
     if(setList <> [])
      then (
      a := hd setList;
      if(a.isOnOpeningWeekDays(d) and a.isInTimeInterval(d))
      then resulList := resulList ^ [a];
      setList := tl setList;
      findRuleOnTime(d, setList, resulList);)
    );

   
  orderByPermission: () ==> ()
  orderByPermission () ==
   while x < y  do
    (BubbleMax());
   
 BubbleMax : () ==> ()
 BubbleMax () ==
  ( dcl z : nat := x;
(*@
\label{findRuleOnTime:117}
@*)
   dcl m:nat := resulList(z).getPriority();
   --find max val in l(x...y)
   for i = x to y do
    if resulList(i).getPriority() > m 
    then (m := resulList(i).getPriority();
      z:= i);
      
    --move max val to index y
      (dcl temp: AccessRule;
      temp := resulList(y);
      resulList(y) := resulList(z);
      resulList(z) := temp;
      y := y-1));
      
  
(*@
\label{orderByPermission:132}
@*)
--public organizeByPriority: seq of AccessRule ==> ()
-- organizeByPriority(resulL) == ();
  
  public filterByUserList: User * seq of AccessRule * seq of AccessRule ==> () --|seq of AccessRule 
   filterByUserList(u, setL, resulL) == (
(*@
\label{BubbleMax:137}
@*)
   dcl a: AccessRule;
   resulList := resulL; 
   setList := setL;
    if(setList <> [])
     then (
     a := hd setList;
     if(a.searchUserInTarget(u))
     then resulList := resulList ^ [a];
     setList := tl setList;
     filterByUserList(u, setList, resulList);)
    );
  
  
  public filterByFacilityList: Facility * seq of AccessRule * seq of AccessRule ==> ()
   filterByFacilityList(f, setL, resulL) == (
   dcl a: AccessRule;
   resulList := resulL; 
   setList := setL;
    if(setList <> [])
     then (
     a := hd setList;
(*@
\label{filterByUserList:158}
@*)
     if(a.searchFacilityInTarget(f))
     then resulList := resulList ^ [a];
     setList := tl setList;
     filterByFacilityList(f, setList, resulList);)
    );

 public createRule: set of User * set of Facility * AccessRule`AccessEvent * nat ==> ()
  createRule(u,f,ae,id) == (
  dcl r : AccessRule := new AccessRule(u,f,ae,id);
  accessPolicy :=  [r] ^ accessPolicy;
  );
  
 public createRule: set of User * set of Facility * Date * Date * AccessRule`AccessEvent * nat ==> ()
  createRule(u,f,d1,d2,ae,id) == (
  dcl r : AccessRule := new AccessRule(u,f,d1,d2,ae,id);
(*@
\label{filterByFacilityList:173}
@*)
  accessPolicy :=  [r] ^ accessPolicy;
  );
  
  public createRule: set of User * set of Facility * Date * AccessRule`AccessEvent * nat ==> ()
  createRule(u,f,d,ae,id) == (
  dcl r : AccessRule := new AccessRule(u,f,d,ae,id);
  accessPolicy :=  [r] ^ accessPolicy;
  );
  
  public getSizeAccessPolicy: () ==> nat
   getSizeAccessPolicy() == (
   return len accessPolicy);
  
  public changeCurrentDate: Date ==> ()
(*@
\label{createRule:187}
@*)
   changeCurrentDate(d) ==(
   currentDate := d;
   );
   
functions

traces
end PhyAccControlSystem
\end{vdmpp}
